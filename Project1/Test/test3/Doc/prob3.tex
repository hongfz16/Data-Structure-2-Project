%!TEX program = xelatex
%!TEX TS-program = xelatex
%!TEX encoding = UTF-8 Unicode

\documentclass[12pt]{article}
\usepackage{amsmath}
\usepackage{geometry}
\usepackage{CJK}
\usepackage{ctex}
\usepackage{graphicx}
\usepackage{booktabs}
\usepackage{pst-node}
\usepackage{pst-text}
\usepackage{amsmath}
\usepackage{amssymb}
\usepackage{listings}
\usepackage{fancyhdr}
\usepackage[colorlinks]{hyperref}
\usepackage{color}
\usepackage{hyperref}
\usepackage{clrscode}
\usepackage{amsthm}


\title{Prob3 Doc}
\author{周展平\\2016013253\\Email: \href{mailto:zhouzp16@163.com}{zhouzp16@163.com}}
\date{\today}
\geometry{left=2cm,right=2cm,top=2cm,bottom=2cm}

\begin{document}
  \maketitle
  \section*{特征:Color Moment}
    矩特征是一类重要的图像特征,颜色矩特征(color moment)是一类常用的矩特征,优势在于维度较低,且能够较为充分表达图像的颜色特征。具体来说,颜色矩特征是一个包含9个维度的向量:
    \begin{equation*}
    color moment = (\mu_1,\mu_2,\mu_3,\sigma_1,\sigma_2,\sigma_3,s_1,s_2,s_3)
    \end{equation*}
    对于一幅$m*n$的图像,3个颜色通道分别占用3个特征,分别为:
    \begin{equation*}
    \mu=\frac{1}{mn}\sum_{i=1}^{m}\sum_{j=1}^{n}x_{ij}
    \end{equation*}
    代表颜色通道的均值(mean),
    \begin{equation*}
    \sigma=\big(\frac{1}{mn}\sum_{i=1}^{m}\sum_{j=1}^{n}(x_{ij}-\mu)^2\big)^{frac{1}{2}}
    \end{equation*}
    代表颜色通道的方差(variance),
    \begin{equation*}
    s=\big(\frac{1}{mn}\sum_{i=1}^{m}\sum_{j=1}^{n}(x_{ij}-\mu)^3\big)^{frac{1}{3}}
    \end{equation*}
    代表颜色通道的斜度(skewness)。
  \section*{不同特征的检索效果比较}
    在本次实验中,我们比较了5种特征的检索结果,分别是:9维、15维、24维的Color Histogram特征,提供的Color Moment特征以及我们自己实现的Color Moment特征。我们采用自己设计的评分策略对检索效果的好坏进行衡量,结果在下面的表格中。
    \begin{table} \caption {特征对检索评分的影响} \centering
       \begin{tabular}{ccc}
        \toprule  %添加表格头部粗线
        特征种类& 得分\\
        \midrule  %添加表格中横线
        Color Histogram(dim=9) & -295.098 \\
        Color Histogram(dim=15) & -258.124\\
        Color Histogram(dim=24) & -275.913\\
        Color Moment(provided) & -2081.56\\
        Color Moment(extracted) & -2434.01\\
        \bottomrule %添加表格底部粗线
        \end{tabular}
    \end{table}
    分析表中数据,可以得到如下结论:(1)Color Histogram特征在检索效果上明显优于Color Moment特征。 (2)维度与Color Histogram特征的检索效果之间没有显著的相关性。 (3)各类特征的得分均小于0,说明检索效果均较差。
\end{document}
%!TEX program = xelatex
%!TEX TS-program = xelatex
%!TEX encoding = UTF-8 Unicode

\documentclass[12pt]{article}
\usepackage{amsmath}
\usepackage{geometry}
\usepackage{CJK}
\usepackage{ctex}
\usepackage{graphicx}
\usepackage{booktabs}
\usepackage{pst-node}
\usepackage{pst-text}
\usepackage{amsmath}
\usepackage{amssymb}
\usepackage{listings}
\usepackage{fancyhdr}
\usepackage[colorlinks]{hyperref}
\usepackage{color}
\usepackage{hyperref}
\usepackage{clrscode}
\usepackage{amsthm}


\title{Prob3 Doc}
\author{周展平\\2016013253\\Email: \href{mailto:zhouzp16@163.com}{zhouzp16@163.com}}
\date{\today}
\geometry{left=2cm,right=2cm,top=2cm,bottom=2cm}

\begin{document}
  \maketitle
  \section*{特征:Color Moment}
    矩特征是一类重要的图像特征,颜色矩特征(color moment)是一类常用的矩特征,优势在于维度较低,且能够较为充分表达图像的颜色特征。具体来说,颜色矩特征是一个包含9个维度的向量:
    \begin{equation*}
    color moment = (\mu_1,\mu_2,\mu_3,\sigma_1,\sigma_2,\sigma_3,s_1,s_2,s_3)
    \end{equation*}
    对于一幅$m*n$的图像,3个颜色通道分别占用3个特征,分别为:
    \begin{equation*}
    \mu=\frac{1}{mn}\sum_{i=1}^{m}\sum_{j=1}^{n}x_{ij}
    \end{equation*}
    代表颜色通道的均值(mean),
    \begin{equation*}
    \sigma=\big(\frac{1}{mn}\sum_{i=1}^{m}\sum_{j=1}^{n}(x_{ij}-\mu)^2\big)^{frac{1}{2}}
    \end{equation*}
    代表颜色通道的方差(variance),
    \begin{equation*}
    s=\big(\frac{1}{mn}\sum_{i=1}^{m}\sum_{j=1}^{n}(x_{ij}-\mu)^3\big)^{frac{1}{3}}
    \end{equation*}
    代表颜色通道的斜度(skewness)。
  \section*{不同特征的检索效果比较}
    在本次实验中,我们比较了5种特征的检索结果,分别是:9维、15维、24维的Color Histogram特征,提供的Color Moment特征以及我们自己实现的Color Moment特征。我们采用自己设计的评分策略对检索效果的好坏进行衡量,结果在下面的表格中。
    \begin{table} \caption {特征对检索评分的影响} \centering
       \begin{tabular}{ccc}
        \toprule  %添加表格头部粗线
        特征种类& 得分\\
        \midrule  %添加表格中横线
        Color Histogram(dim=9) & -295.098 \\
        Color Histogram(dim=15) & -258.124\\
        Color Histogram(dim=24) & -275.913\\
        Color Moment(provided) & -2081.56\\
        Color Moment(extracted) & -2434.01\\
        \bottomrule %添加表格底部粗线
        \end{tabular}
    \end{table}
    分析表中数据,可以得到如下结论:(1)Color Histogram特征在检索效果上明显优于Color Moment特征。 (2)维度与Color Histogram特征的检索效果之间没有显著的相关性。 (3)各类特征的得分均小于0,说明检索效果均较差。
\end{document}
