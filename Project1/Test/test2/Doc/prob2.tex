%!TEX program = xelatex
\documentclass[UTF8]{ctexart}
\CTEXsetup[format={\Large\bfseries}]{section}
\usepackage{fancyhdr}
\usepackage{extramarks}
\usepackage{amsmath}
\usepackage{amsthm}
\usepackage{amsfonts}
\usepackage{tikz}
\usepackage[plain]{algorithm}
\usepackage{algpseudocode}
\usepackage{ctex}
\usepackage{graphicx}
\usepackage{float}
\usepackage{caption}
\usepackage{geometry}

\usepackage{CJK}
\usepackage{booktabs}
\usepackage{pst-node}
\usepackage{pst-text}
\usepackage{amssymb}
\usepackage{listings}
\usepackage{fancyhdr}
\usepackage[colorlinks]{hyperref}
\usepackage{color}
\usepackage{hyperref}
\usepackage{clrscode}


\title{Prob2 Doc}
\author{李帅\\2016013270\\Email: \href{mailto:lishuai16THU@163.com}{lishuai16THU@163.com}}
\date{\today}
\geometry{left=2cm,right=2cm,top=2cm,bottom=2cm}

\begin{document}
  \maketitle
  \section*{Problem2}
  \indent颜色直方图是图像的一种颜色特征。图像的每种颜色都不是单一的,可以由几种单一独立的颜色混合而成。在给定的色彩空间上,将每个像素的颜色沿着不同的通道(每种通道表示一种单一的颜色分量)分离开,统计每种颜色分量的总数或者占所有颜色的比例,用直方图将统计结果表示出来即为颜色直方图。颜色直方图是图像的一种全局特征,他对于图像旋转、缩放、模糊等物理变换并不敏感,因此可以用来衡量和比较不同图像的全局差。但是这种全局特征也会导致像素点间的位置特征丢失,例如几幅整体色调相近的图片,但是颜色的分布完全不一样,这种情况下颜色直方图无法对其进行区分。

  \indent本实验中使用在RGB颜色空间下提取颜色直方图作为特征。RGB颜色空间将颜色分解为红色(Red)、绿色(Green)、蓝色(Blue)三个分量,其中每个分量安亮度可分为256个等级。通过对每个颜色分量的每个亮度值的像素点数进行统计,便可得到3个256维的特征。之后再对每个通道的亮度值进行划分,将一定范围内的像素值归为一类,便能将256维特征压缩到更小的维数。本实验中使用的特征维数是3*3、3*5和3*8。3*3的特征将每个通道的亮度值划分为0-84,85-169,170-255三个范围,其他维数的特征划分与此类似。

  \indent从查询结果所作的曲线图可以看出,随着查询点集的不断增大,维数为9维、15维、24维的颜色直方图作为特征的查询,准确度均不断提高,最终三种维数的查询准确度均稳定在0.25附近。这个准确率并不算太高,对此的分析正如上文所述,颜色直方图是全局的颜色统计特征,不能很好的表达颜色的空间分布,因此对于图形的形状不能很好的分辨,导致了查询准确率较低。

  \indent另一方面,随着查询点集的增大,三个维数的查询准确率始终非常相近,没有拉开差距。对于这个问题的分析是,三个特征看似维度不一,但实际上都是由同样的3*256维向量经过不同的划分而得,因此三个特征包含的信息十分相近,故即使维数不同三者的查询结果也都较为相近。

\end{document}
